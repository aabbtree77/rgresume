\documentclass[a4paper,11pt]{article}

\usepackage[a4paper,
            left=1.05in,
            right=1.05in,
            top=0.95in,
            bottom=1.15in
           ]{geometry}

\usepackage{amsmath}           
\usepackage{tabularx}
\renewcommand{\arraystretch}{1.2}           

%% 1) Load fontspec first
\usepackage{fontspec}

\setmainfont{Inter}[
  UprightFont    = *-Regular,
  ItalicFont     = *-Italic,
  BoldFont       = *-Bold,
  BoldItalicFont = *-BoldItalic,
]

\setmonofont[
  Path           = /usr/local/texlive/2025/texmf-dist/fonts/opentype/adobe/sourcecodepro/,
  Extension      = .otf,
  UprightFont    = *-Regular,
  ItalicFont     = *-RegularIt,
  BoldFont       = *-Bold,
  BoldItalicFont = *-BoldIt,
  Scale          = MatchLowercase,
  %Scale          = 0.9,
]{SourceCodePro}

% 3) Then load polyglossia
\usepackage{polyglossia}
  \setdefaultlanguage{english}
  \setotherlanguages{french,russian,lithuanian,latvian,danish}

% 4) Tell polyglossia explicitly what font to use for each script:
\newfontfamily\englishfont {Inter}
\newfontfamily\frenchfont  {Inter}
\newfontfamily\russianfont {Inter}
\newfontfamily\lithuanianfont{Inter}
\newfontfamily\latvianfont{Inter}
\newfontfamily\danishfont{Inter}



% 3) xurl for proper URL breaking
\usepackage{xurl}


% 4) hyperref for hyperlinks
\usepackage[
  unicode,
  breaklinks=true,
  colorlinks,
  linkcolor=blue,
  urlcolor=blue,
  citecolor=blue
]{hyperref}

% 5) biblatex for bibliography management
\usepackage{csquotes}
\usepackage[backend=biber, style=authoryear, url=true]{biblatex}
\addbibresource{rg.bib}

% 6) microtype for improved typography
\usepackage{microtype}
\microtypesetup{activate=true, expansion=false}
\UseMicrotypeSet[protrusion]{basicmath}


% 7) xstring for string manipulation (protocol stripping)
\usepackage{xstring}

% 8) Restrict URL breaks to hyphens, dots, and slashes
\renewcommand{\UrlBreaks}{\do\/\do\-\do\.}
\renewcommand{\UrlBigBreaks}{}

% 9) Short-URL macro for in-text/tables, removing http:// and https://
\DeclareRobustCommand{\murl}[1]{%
  \href{#1}{%
    \begingroup
      \IfSubStr{#1}{https://}{%
        \StrBehind{#1}{https://}[\ShortUrl]%
      }{%
        \IfSubStr{#1}{http://}{%
          \StrBehind{#1}{http://}[\ShortUrl]%
        }{%
          \edef\ShortUrl{#1}%
        }%
      }%
      \nolinkurl{\ShortUrl}%
    \endgroup
  }%
}

% 10) Redefine url field format for bibliography to match in-text behavior
\DeclareFieldFormat{url}{%
  \mkbibacro{URL}\addcolon\space
  \href{\thefield{url}}{%
    \begingroup
      \IfSubStr{\thefield{url}}{https://}{%
        \StrBehind{\thefield{url}}{https://}[\ShortUrl]%
      }{%
        \IfSubStr{\thefield{url}}{http://}{%
          \StrBehind{\thefield{url}}{http://}[\ShortUrl]%
        }{%
          \edef\ShortUrl{\thefield{url}}%
        }%
      }%
      \nolinkurl{\ShortUrl}%
    \endgroup
  }%
}

\DeclareFieldFormat{url}{%
  \StrBehind{\thefield{url}}{://}[\temp]%
  \href{\thefield{url}}{\nolinkurl{\temp}}%
}


% Force breakable space in volume field
\DeclareFieldFormat[article,inproceedings]{volume}{%
  \bibstring{volume}\space#1%
}

% Redefine inbook macro to allow breaks after booktitle and volume
\renewbibmacro*{in:}{%
  \printtext{%
    \bibstring{in}\intitlepunct
  }%
  \printfield{booktitle}%
  \setunit{\addspace}% Breakable space after booktitle
  \printfield{volume}%
  \setunit{\addcomma\space}% Breakable comma and space
  \printfield{pages}%
}

% Improve line breaking for bibliography
\apptocmd{\bibsetup}{\raggedright \emergencystretch=3em}{}{}

\usepackage{enumitem}

% somewhere in your preamble:
\setlist[enumerate]{%
  label=\arabic*.,            % 1., 2., 3., …
  labelwidth=1.5em,            % reserve enough space for the widest label
  labelsep=0.5em,              % space between label and text
  leftmargin=! ,               % compute leftmargin = labelwidth+labelsep
  align=left,                  % flush the label and block of text
  itemsep=0.5\baselineskip,    % space between items (tweak as you like)
  parsep=0pt
}

\setlist[itemize]{%
  label=•,                 % customize the bullet symbol if you like
  labelsep=0.5em,          % space between bullet and text
  leftmargin=*,            % auto compute left margin (align bullets nicely)
  itemsep=0.5\baselineskip % spacing between items
}

\renewbibmacro{in:}{}
\setlength\bibhang{1em}
\setlength\bibitemsep{0.5em}
\setlength{\emergencystretch}{2em}

\usepackage{textcomp}
% (for A \textrightarrow{} B)

\usepackage{comment}
% (\begin{comment}...\end{comment})

%Do not start new paragraphs with extra space
\setlength{\parindent}{0pt}
\newcommand{\projecturl}[1]{%
  \par\smallskip
  \murl{#1}
  \par\medskip
}


%%%%%%%%%%%%%%%%%%%%%%%%%%%%%%%%%%%%%%%%%%%%%%%%%%%%%%%%%%%%%%%%%%%%%%
%
%
\begin{document}

\begin{center}
\begin{tabular}{ccc}
&\Large \textbf{Curriculum Vitae}&\\
\\
& \textlithuanian{Ramūnas Girdziušas} &\\  
& \textlithuanian{Laisvės} pr. 67-54, Vilnius, LT-07189 &\\
& aabbtree77[at]gmail.com &\\
& \murl{https://aabbtree77.github.io/}
\end{tabular}
\end{center}
%
\section{Education}
%
\begin{tabularx}{\textwidth}{@{}p{3cm}X@{}}
1995-1999:  & BSc (EE), Vilnius Gediminas Technical University.\\
1999-2008:  &  MSc (Tech), DSc (Tech), Helsinki University of Technology.\\
\end{tabularx}
%


\section{Programming Experience (Portfolio)}

\subsection{Web (Full Stack)}

Tools: VPS, ssh, Linux, Porkbun, Go, React, Postgres, SQLite, sqlc, Makefile, Dockerfile, docker-compose.yml, Tailwind, Excalidraw, Mermaid. Also tried: MongoDB Atlas, render.com, MaxMind Geolocation, MERN, Drizzle ORM, Astro.

\subsubsection{Example Project: initials.dev (2026)}

\projecturl{https://initials.dev}

\projecturl{https://github.com/aabbtree77/opt}

A public bulletin board which removes registrations, ads, censorship, and fights bots with proof of work and rate limiters.
Designed and open-sourced backend, frontend, and the VPS infrastructure.

\subsubsection{Other Projects}

\begin{itemize}
\item \murl{https://github.com/aabbtree77/schatzhauser}
\item \murl{https://github.com/aabbtree77/miniguestlog}
\item \murl{https://github.com/aabbtree77/site-redesign-demo}
\end{itemize}

\subsection{Embedded Development}

Tools: ATmega, avr-gcc, ESP32, Ubuntu, routers, MQTT, I2C, libp2p. Wrote firmware for industrial machines (poster printers, box cutters, gym equipment), ensured global device connectivity.

\subsubsection{Example Project: Adast Maxima MS80 (2020 - 2026)}

\projecturl{https://github.com/aabbtree77/adast}

Wrote firmware (C, avr-gcc) for the paper cutting machine, ensuring correctness and precision under heavy factory workloads, with some updates throughout the years. The machine is still in use today (2026).

\subsubsection{Other Projects}

\begin{itemize}
  \item \murl{https://github.com/aabbtree77/esp32-vpn}
  \item \murl{https://github.com/aabbtree77/atmega8-ssd1306-ds18b20-dht22}
  \item  \murl{https://github.com/aabbtree77/digispark-attiny85-command-line}
\end{itemize}

\subsection{3D}

Tools: OpenGL, GLSL, C++, Nim, Go, Blender, GLFW, GLTF, GTX 760. OpenFOAM, ParaView, ProActive PACA Grid (INRIA). Shape analysis (MDS, geodesics).

\subsubsection{Example Project 1: Volumetric Lighting}

\projecturl{https://github.com/aabbtree77/twinpeekz}

A complete rendering pipeline with one of the most beautiful special effects - volumetric lighting.

\subsubsection{Example Project 2: Automotive Optimization}

\projecturl{http://hal.archives-ouvertes.fr/hal-00723427}

A large-scale engineering optimization project involving Navier–Stokes simulations, grid computing, meta-optimization, kriging, Bayesian methods, and Monte Carlo analysis, executed end-to-end on cloud infrastructure in collaboration with academic and industrial partners.

\subsubsection{Other Projects}

\begin{itemize}
  \item \murl{https://github.com/aabbtree77/twinpeekz2}
  \item \murl{https://github.com/go-gl/gl/issues/128}
  \item \murl{https://diglib.eg.org/handle/10.2312/3dor.20141044.009-015}
  \item \murl{https://github.com/doitsujin/dxvk/issues/3789}
  \item \murl{https://github.com/Max1412/Graphics2/issues/24}
  \item \murl{https://github.com/godlikepanos/anki-3d-engine/issues/59}
  \item \murl{https://github.com/aabbtree77/chateaudanet}
\end{itemize}

\subsection{Data Analysis}

Tools: Matlab, Scilab, Python. Kriging, Bayes, CMA-ES.   

\subsubsection{Example Project: MNIST Digits}

\projecturl{https://github.com/aabbtree77/MNIST-0.17}

This repo contains two world records in the MNIST digit recognition competition. One is by me within classics (log-Gabors with max pooling and kriging) - 0.29\%, and the other is a confirmed absolute world record with deep learning by Jonas Matuzas - 0.17\%. 

\subsubsection{Other Projects}

\begin{itemize}
  \item \murl{https://github.com/aabbtree77/imdb-sqlite-queries}
  \item \murl{https://github.com/aabbtree77/uci-marketing-analysis-cart}
  \item  \murl{https://aabbtree77.github.io/tensors}
  \item \murl{https://github.com/polydojo/jispy/issues/1}
\end{itemize}

\section{Employment}
%
\begin{tabularx}{\textwidth}{@{}p{3cm}X@{}}
01.02.2014-Now & Self-Funded. \\

01.02.2013-31.01.2014 &
Faculty of Informatics at the University of Lugano, Via Giuseppe Buffi 13, CH-6904 Lugano, Switzerland. \\

01.09.2011-31.12.2012 &
DEMO, École Nationale Supérieure des Mines de Saint-Étienne, 158 cours Fauriel, 42023 Saint-Étienne, CEDEX2, France. \\

01.09.2008-31.08.2009 &
Department of Neurobiology at the University of California, Los Angeles, 73-235 CHS, Los Angeles, CA 90095-1763. \\

01.05.2000-31.07.2008 &
Faculty of Information and Natural Sciences at Helsinki University of Technology, P.O.Box 5400 (Konemiehentie 2), FI-02015, Finland. \\
\end{tabularx}

\section{Publications}
\label{sect:pubs}
%
\begin{enumerate}
\item \fullcite{boscaini2013}.
\item \fullcite{rg2012a}.
\item \fullcite{rg2012b}.
\item \fullcite{rg2012c}.
\item \fullcite{rg2012d}.
\item \fullcite{rg2012e}.
\item \fullcite{rg2009}.
\item \fullcite{rg2008}.
Advisors: E. Oja and J. Laaksonen (Helsinki Univ. of Tech.). External examiners: S. Siltanen (Tampere Univ. of Tech.) 
and K. Ruotsalainen (Univ. of Oulu). Opponents: S. Siltanen and A. Gotchev (Tampere Univ. of Tech.).
\item \fullcite{rg2007a}.
\item \fullcite{rg2007b}.
\item \fullcite{rg2005a}.
\item \fullcite{rg2005b}.
\item \fullcite{rg2005c}.
\item \fullcite{rg2005d}.
\item \fullcite{rg2005e}.
\item \fullcite{rg2004}.
\item \fullcite{aksela2003}.
\item \fullcite{aksela2002}.
\item \fullcite{rg2001a}.
\item \fullcite{rg2001b}.
\end{enumerate}

%
\section{Talks}
%
\begin{itemize}
    \item Presentations: ICONIP 2001, 2004; ICAPR 2005; ICANN 2005; OMD2 meetings (Paris 2011, 2012).
    \item Posters: SCIA 2005; IJCNN 2005; ECML 2005; ICCV 2007; ACCV 2007.
    \item Invited talk at the DIKU group, the University of Copenhagen, 2008.
\end{itemize}
%
\section{Awards}
%
\begin{tabularx}{\textwidth}{@{}p{3cm}X@{}}
        01.09.1999-01.06.2000 & Nokia scholarship (Natalia Kantola, Robertas Tamulevičius).\\
        01.01.2002-01.07.2006 & ComMIT grant (postgraduate studies).\\
        09.11.2005-09.15.2005 & ICANN travel grant.\\
        07.01.2006-12.01.2006 & HUT scholarship.\\
        01.22.2008-01.24.2008 & Travel grant (Prof. Mads Nielsen).\\
        04.03.2008: & Doctoral thesis award: “Pass with Distinction”.
\end{tabularx}
%
\section{Recommendation Letters}
%
\begin{itemize}
\item Prof. Erkki Oja, Konemiehentie 2, P.O. Box 5400, 02015 HUT, Finland. \\
Email: Erkki.Oja@hut.fi, phone: +358-9-451 3265, fax: +358 9 451 3277.
\item Doc. Jorma Laaksonen, Konemiehentie 2, P.O. Box 5400, 02015 HUT, Finland. \\
Email: Jorma.Laaksonen@hut.fi, phone: +358-9-451 3269, fax: +358 9 451 3277.
\item Prof. Dario Ringach, Dep. of Neurobiology, UCLA, Los Angeles, CA 90095-1563. \\
Email: dario@ucla.edu, phone: +310 825 6606.
\item Doc. Rodolphe Le Riche, DEMO, 158 cours Fauriel, 42023 Saint-Étienne, CEDEX 2.\\ 
Email: leriche@emse.fr, phone: +336 31 262 678.
\item Prof. Michael M. Bronstein, SI-109, Via Giuseppe Buffi 13, 6904, Lugano, Switzerland.\\
Email: michael.bronstein@usi.ch, phone: +41 58 666 4120.
\item Saulius Rakauskas (senior electronics engineer at Infovega UAB, retired). Antakalnio 48-16, LT-10304, Vilnius, Lietuva.\\
Email: radiomtx@gmail.com, phone: +370 652 42317.
\item Bronius Grigas (the founder of \textlithuanian{Mituvėlė} UAB). \textlithuanian{Rodūnios} kl. 11, LT-02189, Vilniaus m. sav., Lietuva.\\
Email: bronius@mituvele.lt, phone: +370 686 43642. 

\end{itemize}
%


%
\section{Additional Information}
%
\begin{tabularx}{\textwidth}{@{}p{3cm}>{\raggedright\arraybackslash}X@{}}
		Date of Birth: & 30.12.1977 (Gen X, He/Him).\\
		Citizenship: & Lithuanian (Lithuania, European Union).\\
        Languages: & English (7/10), Russian (8/10), Lithuanian (9/10).\\
        Driver's Licence: & B.\\
        Sports: & 2318 (hyperboloid777 @ lichess), Kyokushin 2nd belt (1993).\\
        Erdős Number: & 5 (Bollobas, Mannila, Bingham, Oja), 4 (Odlyzko, Guibas, Bronstein).\\
   Hobbies: & Classical and modern cinema.         
\end{tabularx}
%

\end{document}
